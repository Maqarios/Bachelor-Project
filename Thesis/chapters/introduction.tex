\chapter{Introduction}
\label{chap:introduction}
% ================================================================
Disabled people are those with impairment that can be mental, physical, intellectual, etc. However, the main target here are those with physical impairment (i.e. \ac{sci}), since their brain is functional to improve their productivity and support their independence.\par
% ----------------------------------------------------------------
The target of this project is to make on-screen keyboard using \ac{bci} instead of using keyboard or mouse. By showing items on screen and measuring brain signals, we can determine what the user wants. A \ac{bci} monitors brain activity using electrodes to detect certain patterns generated by the user's brain. After digitizing \ac{eeg}, signals are pre-processed and passed to \ac{ml} to classify the outcome \cite{article1, inproceedings1, article2, inproceedings2}.\par
% ----------------------------------------------------------------
\ac{eeg} is a type of \ac{erp} initially reported by S. Sutton \cite{article5}. \ac{p300} is the largest component of \ac{eeg} and can be detected during an oddball event. An oddball event happens when the user is presented with sequence of events that can be classified based on frequency of appearance. The categories of this classification are either frequent or infrequent event. The result of stimulation happens 300ms after the event. It is worth noting that, infrequent events can only be detected if they are by surprise.\par
% ----------------------------------------------------------------
Based on Fazel-Rezai et all \cite{article6}, \ac{p300} has proved to be one of the main components of \ac{bci}. Although it was almost abandoned since 1988 till 2000, in recent years, it started to gain momentum among its peers. \ac{p300} has several appealing features such as efficiency, straightforwardness, relatively fast and practically doesn't require training. \ac{p300} has proved that it can be used in wide range of functions and being easy to use in home for disabled people, although it has some concerns such as gaze shifting. However, new paradigms have been introduced as well as new ways of flashing and even improve classification methods \cite{inproceedings1} to enhance the overall experiment.\par
% ----------------------------------------------------------------
In order to identify the different brain patterns without human intervention, \ac{ml} was used. By supplying the model with signals and their corresponding classes, we can identify brain patterns. The \ac{ml} type we are going to use is a supervised one, which requires set of examples and their correct responses.
\clearpage
% ================================================================